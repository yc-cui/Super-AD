\section{Discussion and Conclusion}\label{Discussion and Conclusion}


This paper presents a novel approach to address the identity mapping problem in self-supervised HAD, which is grounded in a unified framework that encompasses three critical aspects: perturbation, reconstruction, and regularization. Through extensive experiments on various hyperspectral datasets, we have demonstrated the effectiveness of our proposed solutions, including superpixel pooling and uppooling, error-adaptive convolution, and online background pixel mining. Our work presents a significant step forward in the field of self-supervised HAD, offering a robust and effective approach to tackle the challenges posed by the IMP. It is hoped that this paper will provide valuable insights and inspire further research for self-supervised HAD.


Building on the proposed framework, future research could explore the proposed three directions to further enhance self-supervised hyperspectral anomaly detection. For instance, developing adaptive perturbation strategies that dynamically adjust superpixel segmentation scales or integrating spectral-spatial masking to enhance anomaly suppression. Additionally, advanced regularization techniques, such as hierarchical background modeling or uncertainty-aware loss functions, could strengthen robustness against false positives. 



